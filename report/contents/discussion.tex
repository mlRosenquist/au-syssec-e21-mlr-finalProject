Retrieving a basic understanding the code and flow of the application did not require much time. This is the result of not utilizing obfuscation when compiling the release build. As per the documentation\cite{android-obfuscation} this does not require a lot of effort. It is done by enabling Proguard in the build.gradle. Extensive logging is aswell present. This might be of no harm but for a release build these should not be present. Lack of obfuscation and logging only gives an adversary an unnecesary edge in finding potential vulnerabilities. External libraries might be the attack vector for an attacker. He might know a vulnerable library and target a application using it. The Dibs and Zendesk library both utilized a embedded view. One added an interface to call Java code from javascript while the other allowed debugging. The interface has been known to be dangerous. If an adversary where to control the Javascript called from the webview, he would be able to perform severe attacks. However, this was patched in sdk version 17 and the minimum sdk version is 21 for the application.

The TLS configuration of the server resulted in a decent rating. Though should the support for TLS 1.0 and TLS 1.1 be deactivated. The targeted api version of the application all support TLS 1.2\cite{android-sslsocket}. The certificate of server is only domain validated. This is the lowest amount of validation. Users know their data is encrypted however they cant truly trust who is at the receiving end. Even though it is easy and cheap to achieve it should atleast be upgraded to organization validated.

In the attempt of the man-in-the-middle attack there was hurdles. The device has to trust the proxy's certificate and a patched apk is required. This might not be realistic to pull of on a user's device. However, an adversary can mount severe attacks if he is to achieve the setup. The attacks he can perform is editing requests/response, DOS or just eavesdropping. All these are extremely critical and combined violates all policies. In understanding the network traffic an adversary did not need to mount a man-in-the-middle attack, he could just use the application himself. The nature of the traffic is straight forward with the user triggering actions on the server and the server returning some JSON.

It was seen that a token was used for each request. The token was stored in the application own local directory. Furthermore backup of the data was allowed. If an adversary was to retrieve a backup of the data, he would be able to perform requests on the user's behalf. This could potentially be avoided by encrypting the token. The token is retrieved by authenticating in the application. First time users creates their account. Leading to immediate authentication with no email confirmation. This is flaw and results in users being able to create accounts with other persons' emails. Additionally, the passwords are allowed to be quite weak and it is possible to change your password to the same or reusing older ones. This encourages brute force attacks. After having authenticated in the application the token is stored and used in further sessions. Therefore if the phone is unlocked anyone can access the application. This could be resolved by introducing "something you are" authenticity in the form of finger or face scan. Currently the application only uses "something you know" and keeps the user authenticated after first sign in.  

The application is not as intrusive as witnessed from other horror stories. The permissions is quite ordinary and expected. However, location permission handling could be improved. There should be transparency in what the permission is used for and it should be possible to limit it for sessions. Advertisements are shown in the application and they are personalized by a range of trackers. These trackers retrieve continuous data from the users' behaviour. This is sort of expected from nowadays applications. But it should be described in a Privacy Policy or Terms and Condition accesible for the user. These are nowhere to be found in the application.