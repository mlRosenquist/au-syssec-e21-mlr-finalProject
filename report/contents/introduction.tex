This report will document a preliminary security analysis of a mobile application. The application will be looked at from four various aspects: \textit{Software Security}, \textit{Network Security}, \textit{Authentication} and \textit{Privacy}. Four areas that is all key in keeping an application secure. The system that is analyzed is \textit{GulOgGratis} which is an online marketplace targeting the danish market. Establishing such a community requires storing a significant amount of data where some will have a sensitive nature. Users create their accounts and publishes an item they want to sell. Others then communicate with the seller and come to an agreement. GulOgGratis integrates with a range of other services such as; NemID, Parcel Services, Analytical Services, Social Platforms and Ad Services. The marketplace can be accessed either as a website or mobile application. The mobile application is both available for iOS and Android. This report will solely focus on the android application and it's dependencies.  

\subsection{Threat Modelling}
 Modelling the threats for the application we look at four key points. \textit{Threat Model},  whom are we protecting against. \textit{Attack Surface}, which parts of the system can be attacked. \textit{Policies}, which security properties are to be provided. \textit{Mechanisms}, how are the security properties provided. The adversaries are competitors looking for a competetive edge, insiders and criminals with malicious intents. Their attack vectors are the application binaries, access to users accounts, network traffic and the data stored in various cloud services. The threat modelling is as follows:\\       


\noindent\textbf{Threat Model:} competitors, insiders, criminals

\noindent\textbf{Attack Surface:}
\begin{itemize}
    \item Application: decompilation, evil version and broken access 
    \item Network Traffic: mitm, sniffing, manipulation and dos
    \item Integrations: oversharing, data loss and leakage
\end{itemize}

\noindent\textbf{Expected Policies:} data authenticity/confidentiality/integrity, privacy and availability 

\noindent\textbf{Expected Mechanisms:} code obfuscation, access control, secure networking (pinning / certificate transparency), use of trusworthy integrations, permission handling\\

The mentioned policies and mechanisms are what is expected from the application. These will be further evaluated after the security analysis in the discussion. A plausible attack could be an attacker getting access to an user's account. This could be by investigating decompilated code and then knowing which traffic to sniff when the user tries to authenticate. If the attacker is able to mount such a man-in-the-middle attack he is capable of retrieving all data of the user and perform actions on the user's behalf. Another attack could revolve around the user's privacy. Users actions in the application is most likely to be heavily monitored. As this will be stored in the cloud, it creates a privacy concern for the user, if there is some way these actions can be linked to him. Then this can possibly be accessed by some insider, and he might be able to see sensitive data such as his location.      
